\documentclass[a4paper,10pt]{report}

\usepackage[utf8]{inputenc}

\title{Pràctica 1 d'Intel·ligència Artificial}
\author{Josep Maria Olivé Fernández,
        Pol Monroig,
        Yaiza Cano}
\date{20 de Març del 2020}

\begin{document}

	\maketitle

	\chapter*{Introducció}

	La documentación deberá incluir:
	La descripción/justificación de la implementación del estado
	La descripción/justificación de los operadores que habéis elegido
	La descripción/justificación de las estrategias para hallar la solución inicial La descripción/justificación de las funciones heurísticas
	Para cada experimento:
	• Condiciones de cada experimento
	• Resultados del experimento
	• Qué esperabais y qué habéis obtenido
	• Comparaciones
	• Comentarios adicionales que os parezcan adecuados
	Comparación entre los resultados obtenidos con Hill Climbing y Simulated Annealing (no olvidéis explicar cómo habéis ajustado los parámetros para este último algoritmo).
	Respuestas razonadas a las preguntas del enunciado.



	INFORMACIÓ DEL TREBALL DE INNOVACIÓ
	Breve descripción del tema que habéis escogido (no deberíais necesitar más de 3 líneas)
	Reparto del trabajo entre los miembros del grupo. Básicamente quién se ha encargado de la búsqueda de la información para desarrollar cada apartado del trabajo
	Lista de referencias que hayáis encontrado, indicando para que apartados del documento son relevantes y la fecha en la que accedisteis a la referencia.
	Dificultades que hayáis encontrado a la hora de buscar la información necesaria para el trabajo


	\chapter*{Experiment 1}

		En aquest experiment el nostre objectiu és descobrir quin dels nostres operadors és el més efectiu amb igualtat de condicions. Aquestes condicions estan detallades a continuació.\newline

		\section*{Condicions del experiment:}
		Nº d'usuari: 200\newline
		Nº màxim de peticions per usuari: 5\newline
		Nº servidors: 50\newline
		Nº mínim de replicacións: 5\newline
		Algorisme: Hill Climbing\newline
		Estrategia d'inicialització: inicialització random (initialState2)\newline

		\section*{Resultats del experiment:}
		Volem veure quin conjunt d'operadors dona millor resultat















\end{document}